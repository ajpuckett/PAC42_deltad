The following is a list of intended contributions to the proposed experiment from the collaborating institutions. 
\begin{itemize}  

\item The Los Alamos group has played a lead role
  in transverse nucleon spin structure experiments including the JLab Hall A
  transversity experiment E06-010. The LANL group initiated the effort to develop the physics motivation for the proposed experiment by collaborating with leading
  theorists working in the field of nucleon spin structure to characterize the physics
  impact of the proposed measurements in depth.  The LANL group intends to seek DOE as well as LANL internal fundings to support work on RICH hardware, hadron-arm trigger electronics as well as physics developments of this experiment.  While funding permits, the LANL group will commit a postdoc fellow and one-half staff FTE to this experiment, similar to the level committed to the Hall A Transversity Experiment (E06010).    

\item  The University of Connecticut (UConn) Group will take responsibility for the preparation of the RICH detector that will be used in SBS for the proposed experiment and the approved transversity experiment E12-09-018, including testing of the PMTs and optical components of the RICH. UConn will also make major contributions to Monte Carlo simulations and software development for both reconstruction and physics analysis of all the SBS experiments. The UConn group intends to put at least one Ph.D. thesis student on the proposed experiment.

\item The UVa group will take responsibility for the construction and operation
  of the high-polarization high-luminosity $^3$He target, which is also a major part
  of an approved GEN experiment.
  The UVa group will also take responsibility for the construction of the GEM-based tracker
  in the Super Bigbite Spectrometer and its operation. 

\item The CMU group will use their expertise in calorimeters to implement the hadron calorimeter
  and the beam line magnetic shielding, both of which are also required
  in the GEP5 experiment E12-07-109. The CMU group also plays a lead role in
  the approved $G_M^n$ experiment using the SBS.
  
\item The INFN group is committed to the realization of the approved $G_{Ep}$ experiment and 
  to providing a GEM tracker for BigBite, as well as taking the lead in its operation and support. 
  They will also have a significant role in the implementation of the RICH detector.   
  The source of funding for this group is INFN.

\item The Hall A collaborators will take responsibility for the infrastructure associated with
  the 48D48 magnet, which will be used in both this and the three
  approved SBS-related Form-Factor experiments.
\end{itemize}
%  \section{Los Alamos National Laboratory}
