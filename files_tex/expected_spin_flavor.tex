
In this Charpter, we present expected results on nucleon-spin flavor decomposition according various methods described in Chapter 2, and 
the corresponding statistical uncertainties.  We also address major sources of systematical uncertainties.

\section{Systematic uncertainties on asymmetry $A_{1He}^h$ and $A_{1n}^h$}

Knowledge of target polarization and dilution factor dominates the systematic uncertainty of $A_{1n}^h$.
The effects of radiative corrections will be treated in a Monte Carlo simulation
following the procedures of the HERMES analysis~\cite{hermes2002}, which found that the 
systematic uncertainties introduced by this procedure are negligible.  
Kinematic smearing will also be treated following the procedure of the HERMES analysis. 

\vspace{5.0mm}
\noindent ~\hspace{0.25cm} ~Major systematic uncertainties on $A_{1He}^h$:  \\
\noindent ~\hspace{0.25cm} ~Target polarization $\delta P_T/P_T$:  \hfill           $\pm 2.5\%$  relative \\ 
\noindent ~\hspace{0.25cm} ~Beam polarization $\delta P_B/P_B$:    \hfill           $\pm 2.0 \%$ relative \\
\noindent ~\hspace{0.25cm} ~Helicity correlated beam charge uncertainty $\delta (Q_+/Q_-)$:   \hfill  $\ll 10^{-4}$ absolute \\
\noindent ~\hspace{0.25cm} ~Radiative correction and kinematic smearing:                \hfill           $\pm 1.5 \%$  relative \\
\noindent ~\hspace{0.25cm} ~Knowledge of $R$ and correction from $A_{\perp}$:                \hfill           $\pm 1.5 \%$  relative \\
\vspace{-3.0mm}
{\bf 
\noindent \hspace{0.5cm} Total systematic uncertainty on $A_{1He}^h$  \hfill          $\pm 3.8 \%$  relative } \\

\vspace{0.0mm}
\noindent ~\hspace{-0.4cm} ~Extra systematic uncertainties involved in extracting $A_{1n}^h$:  \\
\noindent ~\hspace{0.25cm} ~Dilution factor $\delta f/f$:        \hfill           $\pm 2.5 \%$  relative \\
\noindent ~\hspace{0.25cm} ~Effective neutron polarization in $^3$He $\delta P_n/P_n$:  \hfill           $\pm 4.2\%$  relative \\ 
%\noindent ~\hspace{0.25cm} ~$\pi N$ final state interaction:  \hfill           small \\ 
\vspace{-3.0mm}
{\bf 
\noindent \hspace{0.5cm} Systematic uncertainty on $A_{1n}^h$ (exp.+theory):  \hfill          $\pm 6.2 \%$  relative } \\

\section{Expected results from NLO global fit and impacts to DSSV2008}

\section{Expected results from Leading-Oder ``purity''  method}

\section{Expected results fromLeading-Order Christova-Leader method}
\subsection{Statistical uncertainties of the combined asymmetries  $A_{1He}^{\pi^+ - \pi^-}$ }
The combined asymmetry $A_{1He}^{\pi^+ - \pi^-}$ needs the 
 cross section ratio $r=\sigma_{He}^{\pi^-}/\sigma_{He}^{\pi^+}$ as an extra input:
\begin{equation}
\label{Eq:obs2}
A_{1He}^{\pi^+ - \pi^-}  =  { \Delta \sigma_{He}^{\pi^+} -\Delta \sigma_{He}^{\pi^-} \over
\sigma_{He}^{\pi^+} - \sigma_{He}^{\pi^-} }=
{ A_{1He}^{\pi^+} -  A_{1He}^{\pi^-} \cdot r
\over 1 - r }.
\end{equation}
For this experiment, we have roughly $r=\sigma^{\pi^-}_{He}/\sigma^{\pi^+}_{He}=0.4 \sim 0.8$. 
The error propagation follows:
\begin{equation}
\label{Eq:errobs2}
(\delta A_{1He}^{\pi^+ - \pi^-})^2  = {1 \over (1 - r)^2} \left[ (\delta  A_{1He}^{\pi^+} )^2 + 
r^2 (\delta  A_{1He}^{\pi^-} )^2
+(A_{1He}^{\pi^+} - A_{1He}^{\pi^-} )^2   \cdot {(\delta r)^2  \over (1 - r)^2}  \right].
\end{equation}
The value of $r$ can be reasonably determined to better than 
 $|\delta r|/r \le 2.0 \%$  relatively in this experiment since phase spaces are identical for $\pi^+$ and 
$\pi^-$ measurements.  
 Inside the bracket of Eq.~\ref{Eq:errobs2}, the first term is always larger than the second term, while the third term is in the order of $(0.5\sim 1.5 \times10^{-3})^2$. 
Therefore, the first term dominates in Eq.~\ref{Eq:errobs2}, except in the few very high statistics bins at low-$x$ and low-$z$ in which  $A_{1He}^{\pi^+}$ can be  determined to better than $10^{-3}$. Basically, through Eq.~\ref{Eq:errobs2}, statistical uncertainties of $A_{1He}^{\pi^+ - \pi^-}$ receives an amplification factor  of $1/(1-r)$.

\subsection{Valence quark polarization and their moments}
Following Eq.~\ref{Eq:cl2},  we have  statistical uncertainties:
\begin{eqnarray}
\label{Eq:errcl2}
%&(\Delta u_v)_{LO}&  = {1 \over 5} \left[ \left( 4u_v - d_v)\cdot  A_{1p}^{\pi^+ - \pi^-}
%                 + (u_v + d_v) \cdot A_{1d}^{\pi^+ - \pi^-} \right) \right],  \\
& \delta \left( (\Delta d_v-{1 \over 4} \Delta u_v ) \right)_{LO} &  = {1 \over 4} \left( 7 u_v + 2 d_v \right)  \delta \left(A_{1He}^{\pi^+ - \pi^-} \right).
\end{eqnarray}

Statistical uncertainties of $\Delta d_v-{1 \over 4} \Delta u_v$, according to Leading-Order Christova-Leader spin-flavor decomposition method, are listed in Appendix~\ref{App:spin_flavor}, for each bin of $(x, Q^2, z)$, together with the statistical uncertainties in $A_{1He}^{\pi^+ - \pi^-}$. For most bins of $(x, Q^2, z)$, the combined asymmetry $A_{1He}^{\pi^+ - \pi^-}$ can be determined to better than $1\%$ level.

% list equations here.

When extracting $x( \Delta d_v -{1 \over 4} \Delta u_v )$ 
the knowledge of unpolarized PDF ($\delta q/q \approx \pm 4\%$)  contributes 
to the systematic uncertainties. Theoretical uncertainties 
(PDF and $^3$He to neutron correction) dominate the systematics in $\Delta d_v$, % as listed in Table-\ref{tab:pdf}.

Therefore, over the measured region, we have:
\begin{equation}
 ??? \delta \left[\int_{0.110}^{0.461}(\Delta d_v -{1 \over 4} \Delta u_v)dx \right]_{LO} = \pm 0.023 ~(stat) \pm 0.025~(sys) ???
\end{equation}
Recall that from Eq.~\ref{Eq:bsr2}, if the moment of $\Delta d_v- \Delta u_v$ can be pinned down to $\pm$0.05, 
the moment of polarized sea asymmetry can be constraint to $\delta \left[\int(\Delta \bar{u} - \Delta \bar{d})dx \right] = \pm 0.025$, eight
standard deviations from the prediction of Chiral Quark soliton model.
 
 This result on the valence quark moment is to be compared with the COMPASS deuteron results~\cite{compass2007} at $Q^2=10.0$ GeV$^2$:
\begin{equation}
  \left[\int_{0.006}^{0.7}(\Delta u_v + \Delta d_v)dx \right]_{LO} =  0.40 \pm 0.07 ~(stat) \pm 0.06~(sys) 
\end{equation}


\subsection{??? Need doble-check ??? Statistical and systematic uncertainty $\Delta \bar{u} -\Delta \bar{d}$}
To further extract $\Delta \bar{u}-\Delta \bar{d}$ according to Eq.~\ref{Eq:cl4}, 
knowledge of $\Delta u_v - \Delta d_v$ is needed.  
Of course, the best option is to perform measurements on three different polarized 
targets (proton, deuteron and $^3$He) within the same experimental set up, such that
consistency checks are possible to set limits on systematic uncertainties of $\Delta u_v - \Delta d_v$.

Given the large installation overhead of handling two types of 
polarized targets, we propose here to follow the ``second best'' option.  
The neutron ($^3$He) data from this experiment, or $\Delta d_v - {1 \over 4} \Delta u_v$,
can be combined with the world data on polarized proton target, 
%especially from the planned CLAS12 measurement,
to obtain the best knowledge of $\Delta u_v - \Delta d_v$.

Assuming the world data of $\Delta u_v$ will reach the similar statistical and systematic uncertainties of this experiment, 
Table~\ref{tab:pdfstat} lists the uncertainties on $x(\Delta \bar{u}-\Delta \bar{d})_{LO}$. Uncertainties due to the existing knowledge~\cite{lit:bbfit,xiaochao} 
of inclusive $g_1^p(x, Q^2)$ and $g_1^n(x, Q^2)$ 
($\delta g_1^p = 0.0059$, $\delta g_1^n=0.0057$) are also
included. We note that improvements from the inclusive data set of CLAS12 and this experiment 
will further constrain $g_1^p(x, Q^2)$ and $g_1^n(x, Q^2)$. 

\section{??? Need double-check??? Other systematic uncertainties}

\subsection{Effective nucleon polarization in $^3$He} \label{ch5:he3model}
Effective nucleon polarization in $^3$He for deep-inelastic scattering gives:
\begin{eqnarray}\label{equ:he3-g1n}
 g_1^{^3 {He}} &=& P_ng_1^n+2P_pg_1^p
\end{eqnarray}
where $P_n$($P_p$) is the effective polarization of the neutron 
(proton) inside $^3$He~\cite{theory:PnPp_friar}.
These effective nucleon polarizations $P_{n,p}$ can be
calculated using $^3$He wave functions constructed from N-N interactions,
and their uncertainties were estimated using various nuclear 
models~\cite{theory:PnPp_nogga,theory:PnPp_friar,theory:3Heconv,theory:PnPp_bissey},
giving 
\begin{eqnarray}
&& P_n=0.86^{+0.036}_{-0.02}~~{and}~P_p=-0.028^{+0.009}_{-0.004}~.\label{equ:PnPp}
\end{eqnarray}
%
The small proton effective polarization ($2.8 \%$) causes small 
offsets in the $^3$He asymmetries, compared to that from a 
free neutron.  The uncertainties associated with this small offset are even smaller
when considering that the corresponding proton asymmetries are better known
 and will be improved in the coming years.

At $x=0.110 \sim 0.461$, especially around $x=0.3$, the 
 nuclear EMC effect becomes rather small, as has been demonstrated 
on many different nucleus.

\subsection{$\pi$-$N$ final state interaction }
Since pions carry no spin,  $\pi N$ final state interactions will not introduce 
asymmetries in $A_{1He}^h$. Effect of $\pi$-$N$ final state interaction
will come through the dilution factors. By measuring the leading pions at
$4.3$ GeV/c, where the $\pi$-$N$ total cross sections are reasonably flat, 
effects of FSI are minimized. A detailed $\pi$-$N$ re-scattering calculation~\cite{misak}
confirmed that the modifications to the cross section are rather small at this kinematics.
 

\subsection{Target fragmentation and vector meson production }
 In principle, intermediate $\rho$ production processes are part
 of the fragmentation process and should not be subtracted
 from the SIDIS cross sections. Furthermore, 
 due to the charge conjugation, the effect of intermediate
 $\rho^0$ production is canceled in observables related to
 $\pi^+ - \pi^-$. Therefore, the Christova-Leader method of flavor decomposition
 is not sensitive to $\rho$ production.  
%Calculations of the yield of $(e,e^{\prime}\pi)$  from intermediate $\rho$ production
%have been done in the same Monte Carlo used for SIDIS reaction. The cross
%section for $N(e,e^{\prime}\rho^{\circ})X$ was calculated from a 
%modified version of the formalism used in PYTHIA \cite{Gaskell04}.

 At a high-$z$ setting of this experiment ($z \approx 0.5$), target fragmentation contamination is 
expected to be small, as has been shown by the HERMES LUND based Monte Carlo simulation. In addition, in the
$\pi^+ - \pi^-$ yield target fragmentation contributions are mostly canceled.


\subsection{Corrections from non-vanishing $A^n_{\perp}$ (or $A^n_{LT}$) }   
Since the target polarization is along the beam direction, not exactly along the
virtual photon  direction $\theta_{\gamma^{\star}}$, measurements of
$A_{\parallel}$ should in principle be corrected by a small contribution from $A_{\perp}$
in order to obtain the physics asymmetry $A_{1N}^h$.  
In this experiment, we have  $\sin \theta_{\gamma^{\star}} \approx 0.1$, therefore,
the uncertainty associated with this correction is of the order $0.1 \times (\delta A^n_{\perp})$. 

In the published HERMES and SMC data, the corrections from $A_{\perp}$ were neglected based on
the observation that in inclusive DIS $g_2(x)$ turned out to be rather small.  The residual
effect of non-vanishing $g_2$ (or $A_{\perp}$) in SIDIS has been included in the 
estimation of systematic uncertainties in the HERMES case. The contribution to the fractional 
systematic uncertainties on $A_{1N}^h$ was estimated to be $0.6 \%$ for proton and 
$1.4 \%$ for deuteron.

%In principle, about $10 \%$ of beam time of this experiment should be allocated
%for transverse target runs such that the exact corrections of $A^n_{\perp}$
%can be applied.  However, we chose not to request this extra beam time based
%on the following considerations:

\begin{itemize}
    \item  The leading order contribution in $A_{\perp}$ (or $A_{LT}$ in Mulders' notation)  
           is modulated by an angular dependence of $ \cos(\phi_s - \phi_h)$.
           When a reasonable range of $\phi_h$ is covered,  as in this experiment, 
           the averaged contribution from $A_{\perp}$ will most likely to be washed out.
 
    \item  Aside from the angular modulation, $A^n_{LT}$  was predicted to be at 
           the $10 \%$ level for the proton in bag-model calculations (Mulders, Yuan).
           Assuming $A^n_{LT}$ is at the similar level, the correction to $A_{1n}^h$
           will be at $1 \%$ level for this experiment, much less than the 
           statistical uncertainties.    
             
    \item  The value of $A^n_{LT}$ has been determined in the Hall A Neutron
           Transversity experiment~\cite{E06010_ALT_PRL}, with a in-plane transversely polarized $^3$He target, have provide
            information on $A^n_{LT}$.    Although $A_{LT}$ turned out to be non-zero,  its size is not large~\cite{E06010_ALT_PRL}
            In addition the transverse target run in SBS Transversity will also measure $A^n_{LT}$ to a high precision.
%            Although, $A_{\perp}$ turned out to be non-zero in E06-010, we will spend a few days of beam time 
%           HERMES experiment has already collected data on a transversely polarized
%           proton target with polarized positron beam, from which the  
%           beam-target double-spin asymmetry $A^p_{\perp}$ 
%           can be extracted. $A^n_{\perp}$ is expected to be much smaller than $A^p_{\perp}$.  
\end{itemize}  

Based on the above consideration,  we feel confident that 
even without dedicated transverse target runs the systematic 
uncertainties associated with $A^n_{\perp}$ correction 
will be much less compared to the statistical uncertainties of this experiment. 

