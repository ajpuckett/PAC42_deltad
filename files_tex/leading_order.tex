\section{Leading-order SIDIS asymmetries $A_{1N}^h$}
Following the  short-hand notation of 
Ref~\cite{leader2}, we take the spin-independent cross section as:
\begin{equation}  
 \sigma^{h} (x,z) = \sum_{f} e_f^2 q_f(x) \cdot D_{q_f}^{h}(z),  
\label{eq:fact2}  
\end{equation}  
and the spin-dependent cross section as:
\begin{equation}  
 \Delta \sigma^{h} (x,z) = \sigma^{h}_{++}-\sigma^{h}_{+-}=
 \sum_{f} e_f^2 \Delta q_f(x) \cdot D_{q_f}^{h}(z),  
\label{eq:qfact2}  
\end{equation}  
where $\sigma^{h}_{ij}$ refers to an electron of helicity-$i$ and nucleon of helicity-$j$.
Assuming isospin symmetry and charge conjugation invariance, the number of quark to pion 
 fragmentation functions is   
reduced to three types: the favored ($D_{\pi}^{+}$), the unfavored ($D_{\pi}^{-}$) 
 and the $s$-quark ($D_s^{\pi}$) fragmentation functions:   
\begin{eqnarray}  
 D_{\pi}^{+} & \equiv & D_{u}^{\pi^+} =D_{d}^{\pi^-}=D_{\bar u}^{\pi^-}=D_{\bar d}^{\pi^+}, \nonumber \\ 
 D_{\pi}^{-} & \equiv & D_{u}^{\pi^-} =D_{d}^{\pi^+}=D_{\bar u}^{\pi^+}=D_{\bar d}^{\pi^-}, \nonumber \\
 D_s^{\pi} & \equiv & D_{s}^{\pi^+} =D_{\bar{s}}^{\pi^-}
=D_{\bar u}^{\pi^+}=D_{\bar s}^{\pi^+}.
\label{eq:fragpi}  
\end{eqnarray} 
For the quark to kaon fragmentation functions, the following relations are valid under charge conjugation~\cite{field}:
\begin{eqnarray}  
 D_{K}^{+} & \equiv & D_{u}^{K^+} =D_{\bar{u}}^{K^-}=D_{\bar{s}}^{K^+}=D_{s}^{K^-}, \nonumber \\ 
 D_{K}^{-} & \equiv & D_{u}^{K^-} =D_{\bar{u}}^{K^+}=D_{\bar{s}}^{K^-}=D_{s}^{K^+}, \nonumber \\
 D_d^{K} & \equiv & D_{d}^{K^+} =D_{\bar{d}}^{K^+}=D_{\bar d}^{K^-}=D_{d}^{K^-}.
\label{eq:fragk}  
\end{eqnarray} 

For this experiment, which covers $0.16<x<0.73$,
 we will assume a symmetrical strange quark distribution and polarization
 ($s(x)=\bar{s}(x)$, $\Delta s(x)=\Delta \bar{s}(x)$) and neglect 
 heavy quark contributions.   

\subsection{Spin-dependent and spin-independent cross sections at LO} 
According to Eq.~\ref{eq:fact2}, semi-inclusive  $\pi^{+}$ and $\pi^{-}$ cross section
 on proton and neutron are: 
\begin{eqnarray}  
 9 \sigma_p^{\pi^+} & = & (4 u+\bar{d})D_{\pi}^+ 
 + (4 {\bar u}+ d)D_{\pi}^- + (s+\bar{s}) D_s^{\pi}, \nonumber \\ 
 9 \sigma_p^{\pi^-} & = & (4 u+\bar{d})D_{\pi}^- 
 + (4 {\bar u}+ d)D_{\pi}^+ + (s+\bar{s}) D_s^{\pi}, \nonumber \\ 
 9 \sigma_n^{\pi^+} & = & (4 d+\bar{u})D_{\pi}^+ 
 + (4 {\bar d}+ u)D_{\pi}^- + (s+\bar{s}) D_s^{\pi}, \nonumber \\ 
 9 \sigma_n^{\pi^-} & = & (4 d+\bar{u})D_{\pi}^- 
 + (4 {\bar d}+ u)D_{\pi}^+ + (s+\bar{s}) D_s^{\pi},  
\label{eq:nucleon}  
\end{eqnarray}  
the explicit $x$, $z$, $Q^2$ dependence has been left out to save space whenever
not causing confusion.  
The semi-inclusive  $K^{+}$ and $K^{-}$ cross sections are: 
\begin{eqnarray}  
 9\sigma_p^{K^+} & = & (4 u+ \bar{s})D_{K}^+ 
 + (4 {\bar u}+ s)D_{K}^- + (d +\bar{d}) D_d^{K}, \nonumber \\ 
 9\sigma_p^{K^-} & = & (4 u+ \bar{s})D_{K}^- 
 + (4 {\bar u}+ s)D_{K}^+ + (d +\bar{d}) D_d^{K}, \nonumber \\ 
 9\sigma_n^{K^+} & = & (4 d+ \bar{s})D_{K}^+ 
 + (4 {\bar d}+ s)D_{K}^- + (u +\bar{u}) D_d^{K}, \nonumber \\ 
 9\sigma_n^{K^-} & = & (4 d+ \bar{s})D_{K}^- 
 + (4 {\bar d}+ s)D_{K}^+ + (u +\bar{u}) D_d^{K}. 
\label{eq:nucleonk}  
\end{eqnarray}  
To obtain the spin-dependent cross sections ($\Delta \sigma^h$), one   
replaces the unpolarized quark distributions $q(x)$ in Eq.~\ref{eq:nucleon} and ~\ref{eq:nucleonk} 
with the quark polarization distributions $\Delta q(x)$.

\subsection{The asymmetries expressed in  ``fixed-$z$ purity''} 

For this experiment,   due to the high statistics nature, we can afford to follow the HERMES analysis of  ``purity'' method at every fixed $z$-bins instead of integrating over  $z$ to obtain spin-flavor decomposition, and scrutinizing the $z$-dependent behavior of the extracted parton distributions, as an extra systematic crosscheck.

With the ``fixed-$z$ purity'' method,  we solve for five unknown quantities, taking the notation of $(\Delta s +\Delta \bar{s})/(s+\bar{s}= \Delta s/s$,
 $$X=\{\Delta u/u, \Delta d/d, \Delta \bar{u}/\bar{u},
 \Delta \bar{d}/\bar{d},   \Delta s/s\}$$ first for each kinematic bin of $(x, Q^2,z)$  through measurements of asymmetries $$A=\{ A_{1n}, A_{1n}^{\pi^+}, A_{1n}^{\pi^-},A_{1n}^{K^+},A_{1n}^{K^-} \}$$ with  linear relations  of $A=M\cdot X$.  Notice that for multiple $z$-bins corresponding to the same $(x,Q^2)$, the same inclusive asymmetry $A_{1n}$ will be used.  Once we have clearly verified the $z$-independency of  the extracted polarized parton quantities $X$, we can solve the linear equations once again, combining data of all $z$-bin of the same  $(x,Q^2)$.

We define the ``fixed-$z$ purity'' matrix elements $M_{ij}$ as  the linear coefficients 
in front of $\Delta q/q$ in linear relations of $A=M\cdot X$, as defined in Eq.~\ref{Eq:a1nh}. These coefficients ($M_{ij}$) are obtained from the
unpolarized parton distribution functions and the fragmentation function ratios.   This experiment will have a somewhat lower statistics for SIDIS $\pi^0$ production channel, relative to the charged pion channels, due to the lower efficiency of $\pi^0 \rightarrow 2 \gamma$  reconstruction by the SBS calorimeter.   At this moment, we chose not to include $\pi^0$  asymmetry data in ``fixed-$z$' purity'' LO spin-flavor decomposition.   On the other hand, we include the expected inclusive $A_{1n}$ data points,  they provide similar constrains to $\Delta q/q$ as SIDIS $\pi^0$ channel, but with higher available statistical precisions.  

The  double spin asymmetries, $A_{1N}^h=\Delta \sigma^h/\sigma^h$ can be expressed
at a fixed value of $(x, Q^2, z)$,  we have: 
\begin{eqnarray}
\label{Eq:a1nh}
A_{1n} & = & 
{  \Delta u+ \Delta {\bar u} + 4 ( \Delta d + \Delta \bar{d} ) + 2 \Delta s 
\over u  + {\bar u} + 4(d+\bar{d} )+ 2s },  \nonumber  \\
A_{1n}^{\pi^+} & = & 
{  4\Delta d+ \Delta {\bar u} + \left( 4\Delta {\bar d} + \Delta u \right) 
 \lambda_{\pi} + 2 \Delta s  \xi_{\pi}
\over 4d  + {\bar u} + \left(4{\bar d} + u \right)  \lambda_{\pi} + 
2 s \ \xi_{\pi}},  \nonumber  \\
A_{1n}^{\pi^-} & = & 
{  4\Delta \bar{d}+ \Delta u + \left( 4\Delta d + \Delta \bar{u} \right) 
 \lambda_{\pi} + 2 \Delta s  \xi_{\pi}
\over 4\bar{d}  + u + \left(4 d + \bar{u} \right)  \lambda_{\pi} + 
2 s \ \xi_{\pi}},  \nonumber  \\
A_{1n}^{K^+} & = & 
{  4\Delta d+ \Delta s + \left( 4\Delta {\bar d} + \Delta s \right) 
 \lambda_{K} + (\Delta u+\Delta \bar{u})  \xi_{K}
\over 4d  + s + \left(4{\bar d} + s \right)  \lambda_{K} + 
(u+\bar{u})  \xi_{K}}, \nonumber  \\
A_{1n}^{K^-} & = & 
{  4\Delta \bar{d}+ \Delta s + \left( 4\Delta  d + \Delta s \right) 
 \lambda_{K} + (\Delta u+\Delta \bar{u})  \xi_{K}
\over 4\bar{d}  + s + \left(4d + s \right)  \lambda_{K} + 
(u+\bar{u})  \xi_{K}}. 
\end{eqnarray}

where the fragmentation function ratios are defined as:
\begin{eqnarray}
\lambda_{\pi}(z)  & = & D_{\pi}^-(z)/D_{\pi}^+(z), 
\hspace{1.0cm} \xi_{\pi}(z)=D_{s}^{\pi}(z)/D_{\pi}^+(z), \nonumber \\
\lambda_{K}(z) & = & D_{K}^-(z)/D_{K}^+(z),
\hspace{1.0cm} \xi_{K}(z)=D_{d}^{K}(z)/D_{K}^+(z).
%\label{Eq:dratio}
\end{eqnarray}  

Therefore, through Eq.~\ref{Eq:a1nh}, we can read off matrix elements of $M_{ij}$, for example:
 \begin{eqnarray}
\label{Eq:mij}
M_{11} & = & 
{ u 
\over u  + {\bar u} + 4(d+\bar{d} )+ 2s },  \nonumber  \\
M_{21} & = & 
{  u \cdot  \lambda_{\pi} 
\over 4d  + {\bar u} + \left(4{\bar d} + u \right)  \lambda_{\pi} + 
2 s \ \xi_{\pi}},  \nonumber  \\
M_{31} & = & 
{   u 
\over 4\bar{d}  + u + \left(4 d + \bar{u} \right)  \lambda_{\pi} + 
2 s \ \xi_{\pi}},  \nonumber  \\
M_{41} & = & 
{   u \cdot  \xi_{K}
\over 4d  + s + \left(4{\bar d} + s \right)  \lambda_{K} + 
(u+\bar{u})  \xi_{K}}, \nonumber  \\
M_{51} & = & 
{  u \cdot  \xi_{K}
\over 4\bar{d}  + s + \left(4d + s \right)  \lambda_{K} + 
(u+\bar{u})  \xi_{K}},  etc. 
\end{eqnarray}


\section{Systematic uncertainties in Chritova-Leader LO method introduced by SU(2) symmetry 
violations }

This section was originally provided to PAC34 to address questions raised by Dr. N. Makins  on systematic uncertainties introduced by 
SU(2) symmetry violations in fragmentation functions 
when extracting  $\Delta d_v - {1 \over 4} \Delta u_v$ and $\Delta u_v - \Delta d_v$, $\Delta \bar{u} - \Delta \bar{d}$
when combined with (future) JLab-12 GeV  proton target data. 

%\begin{eqnarray}  
% D_{\pi}^{+} & \equiv & D_{u}^{\pi^+} =D_{d}^{\pi^-}=D_{\bar u}^{\pi^-}=D_{\bar d}^{\pi^+}, \nonumber \\ 
% D_{\pi}^{-} & \equiv & D_{u}^{\pi^-} =D_{d}^{\pi^+}=D_{\bar u}^{\pi^+}=D_{\bar d}^{\pi^-}
%\label{eq:fragpi2}  
%\end{eqnarray} 

Keeping all terms in leading-order, we have:
\begin{eqnarray}  
 9 \sigma_p^{\pi^+} & = & 4 u D_u^{{\pi}^+} + \bar{d} D_{\bar{d}}^{\pi^+} 
   + d D_d^{\pi^+}+ 4 {\bar u} D_{\bar u}^{\pi^+} + (s+\bar{s}) D_s^{\pi}, \nonumber \\ 
 9 \sigma_p^{\pi^-} & = & 4 u D_u^{{\pi}^-} + \bar{d} D_{\bar{d}}^{\pi^-} 
   + d D_d^{\pi^-}+ 4 {\bar u} D_{\bar u}^{\pi^-} + (s+\bar{s}) D_s^{\pi}, \nonumber \\ 
 9 \sigma_n^{\pi^+} & = & 4 d D_u^{{\pi}^+} + \bar{u} D_{\bar{d}}^{\pi^+} 
   + u D_d^{\pi^+}+ 4 {\bar d} D_{\bar u}^{\pi^+} + (s+\bar{s}) D_s^{\pi}, \nonumber \\ 
 9 \sigma_n^{\pi^-} & = & 4 d D_u^{{\pi}^-} + \bar{u} D_{\bar{d}}^{\pi^-} 
   + u D_d^{\pi^-}+ 4 {\bar d} D_{\bar u}^{\pi^-} + (s+\bar{s}) D_s^{\pi}.
\label{eq:nucleon2}  
\end{eqnarray}  

Therefore, cross section differences of $\pi^+ - \pi^-$ are:
\begin{eqnarray}  
 9 \sigma_p^{\pi^+ -\pi^-} & = & 4 u D_u^{\pi^+ -\pi^-} + \bar{d} D_{\bar{d}}^{\pi^+ -\pi^-} 
   + d D_d^{\pi^+ -\pi^-}+ 4 {\bar u} D_{\bar u}^{\pi^+ -\pi^-}, \nonumber \\ 
 9 \sigma_n^{\pi^+ -\pi^-} & = & 4 d D_u^{\pi^+ -\pi^-} + \bar{u} D_{\bar{d}}^{\pi^+ -\pi^-} 
   + u D_d^{\pi^+ -\pi^-}+ 4 {\bar d} D_{\bar u}^{\pi^+ -\pi^-}. 
\label{eq:nucleon3}  
\end{eqnarray}  
Now we will be dealing with 4 flavor non-singlet fragmentation functions instead of 8 regular fragmentation functions. 
We will take two steps in accessing SU(2) violation in fragmentation functions.

\subsection{Case-1: only $u$ and $d$  fragmentation  functions violate SU(2)}
Assuming $\bar{u}$ and $\bar{d}$ fragmentation functions 
still respect SU(2) while $u$ and $d$-quark fragmentation functions violate 
SU(2), i.e.:
\begin{eqnarray}  
 D_{\bar u}^{\pi^+ -\pi^-}/D_u^{\pi^+ -\pi^-} & = & -1, \nonumber \\ 
 D_{\bar d}^{\pi^+ -\pi^-}/D_d^{\pi^+ -\pi^-} & = & -1, \nonumber \\ 
 D_{d}^{\pi^+ -\pi^-}/D_u^{\pi^+ -\pi^-} & = & -1+\epsilon_1.
\label{eq:nucleon4}  
\end{eqnarray}  

\begin{eqnarray}  
 9 \sigma_p^{\pi^+ -\pi^-} & = & 4 (u-\bar{u}) D_u^{\pi^+ -\pi^-}
   + (d-\bar{d}) D_d^{\pi^+ -\pi^-} = 4 u_v D_u^{\pi^+ -\pi^-} + d_v D_d^{\pi^+ -\pi^-} , \nonumber \\ 
 9 \sigma_n^{\pi^+ -\pi^-} & = & 4 d_v D_u^{\pi^+ -\pi^-} + u_v D_d^{\pi^+ -\pi^-}, \nonumber \\ 
 9 \sigma_{2p+n}^{\pi^+ -\pi^-} & = & (8u_v + 4 d_v) D_u^{\pi^+ -\pi^-} + (u_v+2d_v) D_d^{\pi^+ -\pi^-}.
\label{eq:nucleon5}  
\end{eqnarray}  

\begin{eqnarray}
\label{eq:cllo}
&A&\hspace{-0.1cm}_{1p}^{\pi^+ - \pi^-}({\vec p})  =  { \Delta \sigma_p^{\pi^+}-\Delta \sigma_p^{\pi^-} \over
\sigma_p^{\pi^+} - \sigma_p^{\pi^-} }=
{  4\Delta u_v - \Delta d_v + \epsilon_1 \Delta d_v
\over 4u_v - d_v + \epsilon_1 d_v}, \\
&A&\hspace{-0.1cm}_{1He}^{\pi^+ - \pi^-} ({\vec n}+2p) =  { \Delta \sigma_{He}^{\pi^+}-\Delta \sigma_{He}^{\pi^-} \over
\sigma_{He}^{\pi^+}- \sigma_{He}^{\pi^-} }=
{ 4\Delta d_v - \Delta u_v + \epsilon_1 \Delta u_v
\over 7 u_v + 2 d_v + \epsilon_1 (u_v + 2 d_v)}. 
\label{eq:nucleon51}  
\end{eqnarray}

 Assuming a large SU(2) violation  $\epsilon_1= \pm 5 \%$, from Eq.~\ref{eq:nucleon51} the fact that $\epsilon_1 \ne 0$ 
has the following impacts to this experiment. First, the asymmetry $A_{1He}^{\pi^+ - \pi^-}$ has a relative change
of $\epsilon_1 (u_v + 2 d_v)/(7 u_v + 2 d_v) \approx {1 \over 4} \epsilon_1 \approx \pm 1.25 \%$, which is negligible. Second, 
the extracted value of $x(\Delta d_v -{1 \over 4} \Delta u_v)$ has an offset of ${\epsilon_1 \over 4}  x \Delta u_v$. Recall that
$ x \Delta u_v \le 0.3$, therefore, the offset becomes ${\epsilon_1 \over 4}  x \Delta u_v \le 0.3 \epsilon_1 /4 \approx  \pm 0.004 $.
%which is one third of the statistical uncertainties of this experiment.

\subsection{Case-12: only $\bar{u}$ and $\bar{d}$  fragmentation  functions violate SU(2)}
Assuming $\bar{u}$ and $\bar{d}$ fragmentation functions violate SU(2)
while $u$ and $d$-quark fragmentation functions do not, i.e.:
\begin{eqnarray}  
 D_{\bar u}^{\pi^+ -\pi^-}/D_u^{\pi^+ -\pi^-} & = & -1+ \epsilon_2, \nonumber \\ 
 D_{\bar d}^{\pi^+ -\pi^-}/D_d^{\pi^+ -\pi^-} & = & -1 + \epsilon_3, \nonumber \\ 
 D_{d}^{\pi^+ -\pi^-}/D_u^{\pi^+ -\pi^-} & = & -1.
\label{eq:nucleon6}  
\end{eqnarray}  

\begin{eqnarray}  
 9 \sigma_p^{\pi^+ -\pi^-} & = & (4 u_v - d_v) D_u^{\pi^+ -\pi^-} + ( 4 \bar{u} \epsilon_2 - \bar{d} \epsilon_3) D_u^{\pi^+ -\pi^-}, \nonumber \\ 
 9 \sigma_n^{\pi^+ -\pi^-} & = & (4 d_v -u_v) D_u^{\pi^+ -\pi^-} + ( 4 \bar{d} \epsilon_2 - \bar{u} \epsilon_3) D_u^{\pi^+ -\pi^-}, \nonumber \\ 
 9 \sigma_{2p+n}^{\pi^+ -\pi^-} & = & (7u_v + 2 d_v) D_u^{\pi^+ -\pi^-} + (7 \bar{u} \epsilon_2 + 2 \bar{d} \epsilon_3) D_u^{\pi^+ -\pi^-}.
\label{eq:nucleon7}  
\end{eqnarray}  

\begin{eqnarray}
\label{eq:cllo}
&A&\hspace{-0.1cm}_{1p}^{\pi^+ - \pi^-}({\vec p})  =  { \Delta \sigma_p^{\pi^+}-\Delta \sigma_p^{\pi^-} \over
\sigma_p^{\pi^+} - \sigma_p^{\pi^-} }=
{  4\Delta u_v - \Delta d_v + 4 \Delta \bar{u} \epsilon_2 - \Delta \bar{d} \epsilon_3
\over 4u_v - d_v + 4 \bar{u} \epsilon_2 - \bar{d} \epsilon_3}, \\
&A&\hspace{-0.1cm}_{1He}^{\pi^+ - \pi^-} ({\vec n}+2p) =  { \Delta \sigma_{He}^{\pi^+}-\Delta \sigma_{He}^{\pi^-} \over
\sigma_{He}^{\pi^+}- \sigma_{He}^{\pi^-} }=
{ 4\Delta d_v - \Delta u_v + 4 \Delta \bar{d} \epsilon_2 - \Delta \bar{u} \epsilon_3
\over 7 u_v + 2 d_v + 7 \bar{u} \epsilon_2 + 2 \bar{d} \epsilon_3}. 
\label{eq:nucleon71}  
\end{eqnarray}

 Noticed that all terms realted to $\epsilon_2$ and $\epsilon_3$ are associated with sea quark densities.
 Again, assuming a large SU(2) violation  $\epsilon_2= \pm 10 \%$, $\epsilon_3= \pm 10 \%$ , from Eq.~\ref{eq:nucleon71} the fact that $\epsilon_2 \ne 0$ and $\epsilon_3 \ne 0$
has the following impacts to this experiment. First, the asymmetry $A_{1He}^{\pi^+ - \pi^-}$ has a relative change
of $(7 \epsilon_2 \bar{u} + 2 \epsilon_3 \bar{d} / (7 u_v + 2 d_v) \ll  \pm 3 \%$, which is negligible. Second, 
the extracted value of $x(\Delta d_v -{1 \over 4} \Delta u_v)$ has an offset of $\epsilon_2 x \Delta \bar{d}- {\epsilon_3 \over 4}  x \Delta \bar{u}$, 
which is about $0.1 x \Delta  \bar{d}$, rather small compared with statistical uncertainties of this experiment.  

Recall that:
\begin{eqnarray}
\left[\Delta \bar{u}(x) - \Delta \bar{d}(x) \right]_{LO} & = & 3 \left[ g_1^p(x)- g_1^n(x)) \right] 
 - {1 \over 2} (\Delta u_v - \Delta d_v) \vert_{LO}.
% & = & 3 \left[ g_1^p(x)- g_1^n(x)) \right] - {3 \over 8}\Delta u_v (x) \vert_{LO} \nonumber \\ 
% &   & +  {1 \over 8} \left( 7 u_v (x)+ 2 d_v (x) \right) 
%\cdot  A_{1He}^{\pi^+ - \pi^-}(x). 
\end{eqnarray}
For $\epsilon_2= \pm 10 \%$, $\epsilon_3= \pm 10 \%$,  it will introduce $5\% \sim 8 \%$ relative uncertainties to $\Delta \bar{u} -\Delta \bar{d}$. 
